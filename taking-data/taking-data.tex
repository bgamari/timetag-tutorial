\documentclass{article}
\usepackage{verbatim}
\usepackage{graphicx}
\usepackage{url}
\usepackage{hyperref}

\title{Collecting and analyzing FCS and FRET data}
\author{Ben Gamari}

\begin{document}

\maketitle

Here we describe how to perform both collect data with the FPGA
timetagger for both Fluorescence Correlation Spectroscopy and
F\"orster Resonance Energy Transfer experiments.

\section{Fluorescence microscopy}
Fluorescence microscopy is a set of powerful techniques enabling study
of small numbers of molecules. While these techniques vary greatly in
their implementation, the two experiments we will be considering share
some features.

In these experiments, one sends a beam of excitation light though
microscope objective lens, creating a small, brightly illuminated
volume, known as the excitation volume. Fluorescently labelled
molecules will be excited as they enter this region, resulting in an
emission of a different wavelength. This emitted light is collected
through the microscope objective, sent through a pinhole to reduce the
size of the observation volume, and measured with some variety of
light detector.

We will be focusing on two classes of fluorescence microscopy
experiments: Fluorescence Correlation Spectroscopy (FCS) and F\"orster
Resonance Energy Transfer (FRET).

\subsection{FCS}
Fluorescence Correlation Spectroscopy (FCS) is a fluorescence
technique relying on fluctuations in a fluorescence intensity
signal originating from a fluorescently labelled molecule. While many
types of fluctuations can be probed with FCS, two of the most common
are diffusion through the observation volume, and conformational
changes of the molecule. From a raw intensity intensity signal, a time
correlation function allows one to identify timescales on which relevant
dynamics take place.

\subsection{FRET}
F\"orster Resonance Energy Transfer is an energy tranfer mechanism
which will take place between two nearby (up to around 50 nanometers),
electric dipoles.


\section{Collecting data with FPGA timetagger}
The FPGA timetagger records the arrival times of photons from a set of
up to four detectors. The {\tt timetag\_tools} package provides a set
of tools for controlling and acquiring data from the FPGA timetagger.
The {\tt photon\_tools} package then provides a
set of tools to manipulate and analyze this data. The following
discussion assumes that you have both of these packages installed. See \ref{



A FRET experiment will require a sample labelled with both a donor
and an acceptor dye. This is a primary als




\end{document}

