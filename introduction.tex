This is the manual for the FPGA timetagging platform developed in the
Goldner group at University of Massachusetts, Amherst. The FPGA
timetagger is an inexpensive time-tagger designed for
fluorescence spectroscopy experiments. The goal was to produce an
accessible, open-source, and extensible hardware and software
framework for conducting single-molecule sensitive fluorescence
experiments.

In addition to a low-cost piece of hardware, the timetagger includes
an easy-to-use graphical user interface, as well as some basic data
manipulation tools. Moreover, a full complement of analysis and
visualization tools is also available for working with photon
timestamp data from this and other hardware.

The document is structured to bring the user through from
construction to everyday usage of the instrument. While a small amount
of soldering is necessary to build a fully capable unit, the
installation and use of the instrument is intended to require minimal
technical experience. For easiest installation and use, the user is
strongly encouraged to use the instrument on a modern version of
Ubuntu Linux.

